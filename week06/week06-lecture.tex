% !TeX document-id = {a90a2b0a-e08e-44ed-850a-35793bedbf3a}

% !TeX TS-program = PdfLateX

% !BIB program = biber
%\documentclass[handout]{beamer}
\documentclass[compress]{beamer}
\usepackage[T1]{fontenc}
\usepackage{pifont}
\usetheme[block=fill,subsectionpage=progressbar,sectionpage=progressbar]{metropolis} 

\usepackage{minted}

\usepackage{hyperref}
\hypersetup{ %
	pdfborder = {0 0 0},
	colorlinks=true,
}



\definecolor{Purple}{HTML}{911146}
\definecolor{Orange}{HTML}{CF4A30}

% Theme colors are derived from these two elements
\setbeamercolor{alerted text}{fg=Orange}

% ... however you can of course override styles of all elements
\setbeamercolor{frametitle}{bg=Purple}


\usepackage{wasysym}
\usepackage{etoolbox}
\usepackage[utf8]{inputenc}

\usepackage{threeparttable}
\usepackage{subcaption}

\usepackage{tikz-qtree}
\setbeamercovered{still covered={\opaqueness<1->{5}},again covered={\opaqueness<1->{100}}}





\usepackage{listings}

\lstset{
	basicstyle=\scriptsize\ttfamily,
	columns=flexible,
	breaklines=true,
	numbers=left,
	%stepsize=1,
	numberstyle=\tiny,
	backgroundcolor=\color[rgb]{0.85,0.90,1}
}



\lstnewenvironment{lstlistingoutput}{\lstset{basicstyle=\footnotesize\ttfamily,
		columns=flexible,
		breaklines=true,
		numbers=left,
		%stepsize=1,
		numberstyle=\tiny,
		backgroundcolor=\color[rgb]{.7,.7,.7}}}{}


\lstnewenvironment{lstlistingoutputtiny}{\lstset{basicstyle=\tiny\ttfamily,
		columns=flexible,
		breaklines=true,
		numbers=left,
		%stepsize=1,
		numberstyle=\tiny,
		backgroundcolor=\color[rgb]{.7,.7,.7}}}{}


\usepackage[american]{babel}
\usepackage{csquotes}
\usepackage[style=apa, backend = biber]{biblatex}
\DeclareLanguageMapping{american}{american-UoN}
\addbibresource{../literature.bib}
\renewcommand*{\bibfont}{\tiny}

\usepackage{tikz}
\usetikzlibrary{shapes,arrows,matrix}
\usepackage{multicol}

\usepackage{subcaption}

\usepackage{booktabs}
\usepackage{graphicx}

\graphicspath{{../pictures/}}

\makeatletter
\setbeamertemplate{headline}{%
	\begin{beamercolorbox}[colsep=1.5pt]{upper separation line head}
	\end{beamercolorbox}
	\begin{beamercolorbox}{section in head/foot}
		\vskip2pt\insertnavigation{\paperwidth}\vskip2pt
	\end{beamercolorbox}%
	\begin{beamercolorbox}[colsep=1.5pt]{lower separation line head}
	\end{beamercolorbox}
}
\makeatother



\setbeamercolor{section in head/foot}{fg=normal text.bg, bg=structure.fg}



\newcommand{\question}[1]{
	\begin{frame}[plain]
		\begin{columns}
			\column{.4\textwidth}
			\makebox[\columnwidth]{
				\includegraphics[width=\columnwidth,height=\paperheight,keepaspectratio]{../pictures/mannetje.png}}
			\column{.6\textwidth}
			\large
			\textcolor{orange}{\textbf{\emph{#1}}}
		\end{columns}
\end{frame}}

\newcommand{\instruction}[1]{\emph{\textcolor{gray}{[#1]}}}


\title[Computational Communication Science 2]{\textbf{Computational Communication Science 2} \\Week 6 - Lecture\\ »Recommender systems«}
\author[Marthe Möller, Anne Kroon]{Marthe Möller \\ Anne Kroon \\ ~ \\ \footnotesize{A.M.Moller@uva.nl, @MartheMoller \\a.c.kroon@uva.nl, @annekroon} \\}
\date{April, 2022}
\institute[Digital Society Minor, University of Amsterdam]{Digital Society Minor, University of Amsterdam}


\begin{document}
	
	\begin{frame}{}
		\titlepage
	\end{frame}
	
	\begin{frame}{Today}
		\tableofcontents
	\end{frame}

\section{Recommender Systems}

%nowadays, recommender systems are everywhere.
%Recommender systems have been around for several years now, but many people don't realize the extent to which they are ingrained into the experiences we have with technology.
%When you log into Netflix and see movie suggestions, that's their recommender system at work. 
%When you log into Amazon and it suggests products to you, that's their recommender system at work.
%When you fire up Facebook, both the posts that show up in your news feed and the friend recommendations they provide are examples of %their recommender systems at work.

\begin{frame}
	\makebox[\linewidth]{
		\includegraphics[width=\columnwidth,height=\paperheight,keepaspectratio]{../pictures/social_media.jpeg}}
\end{frame}

% recommender systems are everywhere around us.

\begin{frame}{Recommender Systems in Communication Science} 
	\begin{block}{New research questions}
		\begin{enumerate}
			\item<2-> \alert{Political communication and journalism}. E.g., to craft personalized news diets. This may have, however, consequences for the diversity of news diets and for democracy \parencite{Moller2018, Locherbach2018}
			\item<3-> \alert{Organizational and corporate communication}. E.g., applications in the field of hiring and recruitment.
			\item<4-> \alert{Persuasive communication}. E.g., in the health domain--recommendation algorithms for tailored health interventions \parencite{Kim2019}
			\item<5--> \alert{Entertainment communication}. E.g., movie recommenders
				\end{enumerate}
	\end{block}
\end{frame}

\begin{frame}{Recommender Systems} 
	\begin{block}{Two central problems within Recommender Systems}
		\begin{enumerate}
			\item<2-> \alert{Predicting problem}. Typical problem involves a lot of missing data (e.g., user only rated a small subset of movies/news articles/ etc.) How can we deal with missing data? 
			\item<3-> \alert{Ranking problem}.  Given \textit{n} items, can we identify the top \textit{k} items to recommend?
		\end{enumerate}
	\end{block}
\only<4->{These problems are not isolated; rather, they are connected. }
\end{frame}

\begin{frame}{Recommender Systems} 
	\begin{block}{How do recommender systems learn?}
		\begin{enumerate}
			\item<2-> \alert{Explicit user preferences}. Ratings or responses 
			\item<3-> \alert{Implicit user preferences}. E.g., clicks or viewing time
			\item<3-> \alert{Content}. E.g., based on text similarity techniques
		\end{enumerate}
	\end{block}
\end{frame}

\begin{frame}{Recommender Systems} 
	\begin{block}{Types of recommender systems  \parencite{Wieland2021, Locherbach2018, Moller2018}}
\begin{enumerate}
	\item 'Basic' knowledge-base recommender systems
	\item Content-based recommender systems
	\item Collaborative  recommenders 
\end{enumerate}
	\end{block}{\tiny }
\end{frame}

\section{Knowledge-based RecSys}
\begin{frame}{Knowledge-based recommender system} 
	\begin{block}{When to use?}
		\begin{itemize}
			\item<2-> To overcome the \textbf{cold start problem}; when we do not have ratings of individual users. 
			\item<3-> Simple model. It does not rely on user's explicit or implicit ratings, but on specific queries.
			\item<4-> Typical use case: Real-estate. Buying a house is, for most families, a rare/ single event. 
		\end{itemize}
	\end{block}
\end{frame}

% This prints the bibliography on the slide
%\section{An example with some Python code\dots}


\begin{frame}
	\makebox[\linewidth]{
		\includegraphics[width=\columnwidth,height=\paperheight,keepaspectratio]{../pictures/funda}}
\end{frame}

\begin{frame}{Use case: ImDb database}
	\makebox[\linewidth]{
		\includegraphics[width=\columnwidth,height=\paperheight,keepaspectratio]{../pictures/data}}
\end{frame}

\question{What are relevant variables to use in a knowledge-based recommender system?}

\begin{frame}[fragile]{Knowledge-based recommender system}
How can we work with user input without a front-end (such as the website of funda? 
\pause
\\
$\rightarrow$ enter python's native \alert{input()} function.
\pause
\\
\begin{minted}{python}
print("What is your favorite movie genre?")
genre = input()
\end{minted}
\pause
\\
\begin{lstlistingoutput}
what is your favorite movie genre?
[...]
\end{lstlistingoutput}
\end{frame}


\begin{frame}[fragile]{Improving knowledge-based recommender system}
	
\begin{block}{When to use?}
\begin{itemize}
\item It is important to think about ways to make the recommendation relevant for individuals
\pause
\item Do you have more information in your db that make your top-listed recommendations as relevant as possible?
\pause
\end{itemize}
\end{block}

\pause
\begin{minted}
recommend_movies = movies.sort_values('vote_average', ascending=False)
\end{minted}
\end{frame}


\section{Content-based RecSys}

\begin{frame}
\begin{block}{Content-based systems}
	\begin{itemize}
		\item <1-> Recommends items based on user's profiles. 
		\item <2-> Profiles are based on e.g., ratings, and represents user's tasts/ preferences. 
		\begin{itemize}
			\item <3-> For example, how often a user has clicked on, or liked, a movie. 
		\end{itemize}
		\item <4-> Recommendation is based on \alert{similarity} beween items in the content.
		\begin{itemize}
			\item <5-> Content is here: e.g., genre, tags, plot, authors, directors, location, etc. 
		\end{itemize}
	\end{itemize}
\end{block}
\end{frame}

\begin{frame}{Example of a content-based recsys}
	\makebox[\linewidth]{
		\includegraphics[width=\columnwidth,height=\paperheight,keepaspectratio]{../pictures/harry1.png}}
\end{frame}

\begin{frame}{Example of a content-based recsys}
	\makebox[\linewidth]{
		\includegraphics[width=\columnwidth,height=\paperheight,keepaspectratio]{../pictures/harry2.png}}
\end{frame}

%as you can see, it works quite well, but also has its drawbacks.. can be quite obvious.

%using information of the items themselves, rather than the aggregate user data. Even for more advanced recommender systems, baking in some knowledge about the content can be very helpful. 

%The movielens dataset does not give us a lot of information, but it does 

%Let's have a look again at the MovieLens dataset. If we know given user likes a specific genre, the user might also like other genres. In addition, we can use year of release. We can narrow it down further to movies that are close to a specific release data. 
% there is all sort of information availabe here, that we might use. 

\begin{frame}{Use case: ImDb database}
Let's have a look at our use-case again. \\
Are there attributes that you could use to estimate similarity in movies?
	\makebox[\linewidth]{
		\includegraphics[width=\columnwidth,height=\paperheight,keepaspectratio]{../pictures/data}}
\end{frame}

% how can we calculate similarity, just based on genre?
% Cosine is very handy, not only for textule data, but also for different types of numeric data. 

% to make it easier to understand, lets imagine that every movie has only one of the follwoing genres: 
%romance or comedy

\begin{frame}{Similarity between movies}
	\makebox[\linewidth]{
		\includegraphics[width=\columnwidth,height=\paperheight,keepaspectratio]{../pictures/Slide1.png}}
\end{frame}

\begin{frame}{Similarity between movies}
	\makebox[\linewidth]{
		\includegraphics[width=\columnwidth,height=\paperheight,keepaspectratio]{../pictures/Slide2.png}}
\end{frame}

\begin{frame}{Similarity between movies}
	\makebox[\linewidth]{
		\includegraphics[width=\columnwidth,height=\paperheight,keepaspectratio]{../pictures/Slide3.png}}
\end{frame}

\begin{frame}{Similarity between movies}
	\makebox[\linewidth]{
		\includegraphics[width=\columnwidth,height=\paperheight,keepaspectratio]{../pictures/Slide4.png}}
\end{frame}

\begin{frame}{Similarity between movies}
	\makebox[\linewidth]{
		\includegraphics[width=\columnwidth,height=\paperheight,keepaspectratio]{../pictures/Slide5.png}}
\end{frame}

\begin{frame}[fragile]{Let's put this in code!}
\pause
First, create a create a toy dataset.
\pause
\begin{minted}{python}
data = data.sample(6)
data[['title', 'genres']]
\end{minted}
\pause
	\makebox[\linewidth]{
	\includegraphics[width=\columnwidth,height=\paperheight,keepaspectratio]{../pictures/toy}}
\end{frame}


\begin{frame}[fragile]{Feature selection and vectorization}
Let's assume we want to calculate similiarity based on genres. Therefore, we need to vectorize this column. 
	\begin{minted}{python}
tfidf = TfidfVectorizer(stop_words='english')
tfidf_matrix = tfidf.fit_transform(data['genres'])
	\end{minted}
\pause
Note that we use \alert{tfidf vectorizer} here, but we you might opt for a different one. 
\end{frame}



\begin{frame}[fragile]{Calculate similiarity}
Now, let's calculate cosine similarity scores between the genre attributes of the movies
	\begin{minted}{python}
from sklearn.metrics.pairwise import cosine_similarity

sim = cosine_similarity(tfidf_matrix)
	\end{minted}
	\pause
This returns an array of the similiarity scores between each movie and all other movies. 
\end{frame}

\begin{frame}[fragile]{Cosine Similairity}
Let's inspect the output...
	\begin{minted}{python}
print(sim)
[[1.         0.34503493 0.         0.         0.40824829 0.        ]
[0.34503493 1.         0.16581288 0.         0.28171984 0.29130219]
[0.         0.16581288 1.         0.25964992 0.         0.56921261]
[0.         0.         0.25964992 1.         0.44115109 0.4561563 ]
[0.40824829 0.28171984 0.         0.44115109 1.         0.        ]
[0.         0.29130219 0.56921261 0.4561563  0.         1.        ]]
	\end{minted}
\end{frame}


\begin{frame}[plain]
	\makebox[\linewidth]{
		\includegraphics[width=\columnwidth,height=\paperheight,keepaspectratio]{../pictures/cosine_sim}}
\pause
Based on this overview, you may assume that users that like \textit{Avatar}, may be interested in \textit{John Carter}. 
\pause
\\
\\
\footnotesize{If you want to convert output of \alert{cosine\_similarity} to a \alert{df} type of object, see here \url{https://github.com/annekroon/CCS-2/blob/main/week06/cosine_to_df.md} }
\end{frame}

\begin{frame}
\begin{block}{Feature selection and engineering}
\begin{itemize}
	\item Feature engineering is very important here. What qualities or attributes do we want to incorporate?
	\pause
	\item Think about this critically. Instead of using genres or authors, you could think about crafting features yourself, such as topics. 
	\pause
\end{itemize}
\end{block}
\end{frame}


\begin{frame}
\begin{exampleblock}{Benefits}
\begin{itemize}
	\item <1-> Content-based recommender systems can be very efficient...
	\item <2->They are often part of more complex recommender systems that leverage (deep) supervised learning
\end{itemize}
\end{exampleblock}
\begin{alertblock}{Drawbacks}
	\begin{itemize}
		\item <3->Features that are not part of the user profile will be neglected; e.g., if the user does not like Super Hero movies, the recommender system wil never recommend this. 
		\item <4->does not use the power community data. Recommendations may be obvious (e.g., \textit{Harry Potter} recommendation when you like \textit{Lord of the Rings})
	\end{itemize}
\end{alertblock}
\end{frame}

\section{Collaborative  RecSys}
\begin{frame}
	\begin{block}{Collaborative recommender systems}
		\begin{itemize}
			\item <1-> Tries to overcome some of the limitations of content-based systems
			\item <2->Leverages the power of the community, tries to give relevant, but also suprising recommendations. 
		\end{itemize}
	\end{block}
\end{frame}

\section{Group assignment}
\url{https://www.kaggle.com/saurabhbagchi/books-dataset}


\begin{frame}
	\frametitle{References}
	\printbibliography
\end{frame}
	

\end{document}