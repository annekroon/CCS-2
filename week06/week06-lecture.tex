% !TeX document-id = {a90a2b0a-e08e-44ed-850a-35793bedbf3a}
% !TeX TS-program = xelatex

% !BIB program = biber
%\documentclass[handout]{beamer}
\documentclass[compress]{beamer}
\usepackage[T1]{fontenc}
\usepackage{pifont}
\usetheme[block=fill,subsectionpage=progressbar,sectionpage=progressbar]{metropolis} 


\definecolor{Purple}{HTML}{911146}
\definecolor{Orange}{HTML}{CF4A30}

% Theme colors are derived from these two elements
\setbeamercolor{alerted text}{fg=Orange}

% ... however you can of course override styles of all elements
\setbeamercolor{frametitle}{bg=Purple}


\usepackage{wasysym}
\usepackage{etoolbox}
\usepackage[utf8]{inputenc}

\usepackage{threeparttable}
\usepackage{subcaption}

\usepackage{tikz-qtree}
\setbeamercovered{still covered={\opaqueness<1->{5}},again covered={\opaqueness<1->{100}}}


\usepackage{listings}

\lstset{
	basicstyle=\scriptsize\ttfamily,
	columns=flexible,
	breaklines=true,
	numbers=left,
	%stepsize=1,
	numberstyle=\tiny,
	backgroundcolor=\color[rgb]{0.85,0.90,1}
}



\lstnewenvironment{lstlistingoutput}{\lstset{basicstyle=\footnotesize\ttfamily,
		columns=flexible,
		breaklines=true,
		numbers=left,
		%stepsize=1,
		numberstyle=\tiny,
		backgroundcolor=\color[rgb]{.7,.7,.7}}}{}


\lstnewenvironment{lstlistingoutputtiny}{\lstset{basicstyle=\tiny\ttfamily,
		columns=flexible,
		breaklines=true,
		numbers=left,
		%stepsize=1,
		numberstyle=\tiny,
		backgroundcolor=\color[rgb]{.7,.7,.7}}}{}


\usepackage[american]{babel}
\usepackage{csquotes}
\usepackage[style=apa, backend = biber]{biblatex}
\DeclareLanguageMapping{american}{american-UoN}
\addbibresource{../literature.bib}
\renewcommand*{\bibfont}{\tiny}

\usepackage{tikz}
\usetikzlibrary{shapes,arrows,matrix}
\usepackage{multicol}

\usepackage{subcaption}

\usepackage{booktabs}
\usepackage{graphicx}

\graphicspath{{../pictures/}}

\makeatletter
\setbeamertemplate{headline}{%
	\begin{beamercolorbox}[colsep=1.5pt]{upper separation line head}
	\end{beamercolorbox}
	\begin{beamercolorbox}{section in head/foot}
		\vskip2pt\insertnavigation{\paperwidth}\vskip2pt
	\end{beamercolorbox}%
	\begin{beamercolorbox}[colsep=1.5pt]{lower separation line head}
	\end{beamercolorbox}
}
\makeatother



\setbeamercolor{section in head/foot}{fg=normal text.bg, bg=structure.fg}



\newcommand{\question}[1]{
	\begin{frame}[plain]
		\begin{columns}
			\column{.3\textwidth}
			\makebox[\columnwidth]{
				\includegraphics[width=\columnwidth,height=\paperheight,keepaspectratio]{../pictures/mannetje.png}}
			\column{.7\textwidth}
			\large
			\textcolor{orange}{\textbf{\emph{#1}}}
		\end{columns}
\end{frame}}

\newcommand{\instruction}[1]{\emph{\textcolor{gray}{[#1]}}}


\title[Computational Communication Science 2]{\textbf{Computational Communication Science 2} \\Week 6 - Lecture\\ »Recommender systems«}
\author[Marthe Möller, Anne Kroon]{Marthe Möller \\ Anne Kroon \\ ~ \\ \footnotesize{A.M.Moller@uva.nl, @MartheMoller \\a.c.kroon@uva.nl, @annekroon} \\}
\date{April, 2022}
\institute[Digital Society Minor, University of Amsterdam]{Digital Society Minor, University of Amsterdam}


\begin{document}
	
	\begin{frame}{}
		\titlepage
	\end{frame}
	
	\begin{frame}{Today}
		\tableofcontents
	\end{frame}

\section{recommender systems}

\begin{frame}
	\makebox[\linewidth]{
		\includegraphics[width=\columnwidth,height=\paperheight,keepaspectratio]{../pictures/social_media.jpeg}}
\end{frame}

% recommender systems are everywhere around us.

\begin{frame}{Recommender Systems} 
	\begin{block}{How do recommender systems learn?}
		\begin{enumerate}
			\item<2-> \alert{Explicit user preferences}.  e.g., ratings or responses 
			\item<3-> \alert{Implicit user preferences}.  e.g., clicks or viewing time
			\item<3-> \alert{Content}. e.g., based on text similarity techniques
		\end{enumerate}
	\end{block}
\end{frame}

\begin{frame}{Recommender Systems} 
	\begin{block}{Two central problems within Recommender Systems}
		\begin{enumerate}
			\item<2-> \alert{Predicting problem}. Typical problem involves a lot of missing data (e.g., user only rated a small subset of movies/ news articles/ etc.) How can we deal with missing data? 
			\item<3-> \alert{Ranking problem}.  Given \textit{n} items, can we identify the top \textit{k} items to recommend?
		\end{enumerate}
	\end{block}
\only<4->{These problems are not isolated; rather, they are connected. }
\end{frame}

\begin{frame}{Recommender Systems} 
	\begin{block}{Types of recommender systems  \parencite{Wieland2021, Locherbach2018, Moller2018}}
\begin{enumerate}
	\item 'Basic' knowledge-base recommender systems
	\item Content-based recommender systems
	\item Collaborative  recommenders 
\end{enumerate}
	\end{block}
\end{frame}

\section{Knowledge-based recommender system}
\begin{frame}{Knowledge-based recommender system} 
	\begin{block}{When to use?}
		\begin{itemize}
			
			\item<2-> To overcome the \textbf{cold start problem}; when we do not have ratings of individual users. 
			\item<3-> Typical use case: Real-estate. Buying a house is, for most families, a rare/ single event. 
		\end{itemize}
	\end{block}
\end{frame}

% This prints the bibliography on the slide


\section{An example with some Python code\dots}

\begin{frame}[fragile]{A simple bottom-up approach}
\begin{lstlisting}
from collections import Counter
texts = ["I really really really love him, I do", "I hate him"]

for t in texts:
    print(Counter(t.split()).most_common(3))
\end{lstlisting}

\begin{lstlistingoutput}
[('really', 3), ('I', 2), ('love', 1)]
[('I', 1), ('hate', 1), ('him', 1)]
\end{lstlistingoutput}
\end{frame}


\begin{frame}
	\frametitle{References}
	\printbibliography
\end{frame}
	

\end{document}