This course is set-up in the following manner. In a regular week, on the Mondays, we have lectures. Here, key concepts are explained from a conceptual and theoretical perspective, and examples of code implementations is python are provided.  On the Tuesdays, we have lab sessions. During the lab sessions, we generally start with knowledge questions about that week's literature as well as the preceeding lecture. Afterwards, students will work on assignments that are provided. When students have finished the assignments, they can work on the group assignment. The tuturial lecture (Noon will walk around, and will help students with issues they encounter). If multiple students are encountering the same issues, these problems will be discussed plenarily. Active participation in the tutorial meetings is needed to ensure that students understand the materials, and can work on the (group) assignments.

\section*{Week 1: Course introduction \& Working with textual data}

\subsection*{Monday, 4--4: Lecture}
\textsc{\ding{52} Read this manual and inspect the course Canvas page.}\\
\textsc{\ding{52} Make sure your computer is ready to use for the course }\\
\textsc{\ding{52} Read in advance: \cite{Hirschenberg2015}.} \\
During the first meeting, we will introduce the course: we explain the course goals and policies. We also take a first look into using computational methods for communication science by discussing what we can learn from analyzing texts.


\subsection*{Tuesday, 5--4. Lab session}
During our first tutorial lecture, we will take what we discussed in the lecture and use this to analyze text.

\section*{Week 2: Text as data}

\subsection*{Monday, 11--4: Lecture}
\textsc{\ding{52}Read in advance: chapter 10: Text as data \cite{van_atteveldt_computational_2022}.} \\
\textsc{\ding{52}Read in advance: \cite{Boumans2016}.} \\

During this meeting, we will continue with were we left off in the first week. Using some of the techniques of week 1, we will explore some simple, deductive approaches and reflect on their advantages and disadvantes. As an alternative, we will reflect on some of the benefits of inductive, bottom-up approaches. To this aim, we will discuss Bag-of-Words (BOW) representations of textual data, and discuss multiple ways of transforming text into matrices. 

\subsection*{Tuesdag, 12--4: Lab session}
In this week's lab session, we will practice with some simple top-down and bottom-up algorithm for text analysis. In order to prepare ourselves for more advanced bottom-up techniques in week 4, we will start practicing with transforming textuel to matrices, using  \texttt{sklearn}'s vectorizers. 

\section*{Week 3: Start of the group assignment}

\subsection*{Monday, 18--4: No lecture--UvA teaching free day}

\subsection*{Tuesday 19--4: Lab session}
During this tutorial meeting, we will form small groups of 3-4 students and start working on the group assignment. Specifically, we will make a start with exploring the dataset as provided, inspect relevant variables, and start working on cleaning the dataset and division of tasks. 

\section*{Week 4: Bottom up approaches to text analysis}

\subsection*{Monday, 25--4: Lecture}
\textsc{\ding{52}Read in advance chapter 11: Automatic analysis of text \cite{van_atteveldt_computational_2022}.} \\
\textsc{\ding{52}Read in advance: \cite{Brinberg2021}.} \\

In this meeting, we will discuss some prominent techniques to analyses text using bottom up techniques. Specifically, we will discuss latent dirichlet allocation, a populair approach to detect topics in textual data. In addition, we will focus on different ways to measure similarity between texts, using cosine and soft cosine similarity. 

\subsection*{Tuesdag, 26--4: Lab session}
During this tutorial meeting, we will practice with the concepts and code as introduced in Monday's lecture. We will try to measure the similarity between sets of documents using different approaches. In addition, we will practice with a simple LDA model. \\
In case you understand all concepts, and finished the in-class assignments, you can continue working on the group assignment. 

\section*{Week 5: Taking a break}

\subsection*{Monday, 2--5: No Lecture -- UvA teaching free week}

\subsection*{Tuesday, 3--5: No Lab session -- UvA teaching free week}

\textsc{\ding{52} Take a break}\\

\emph{Note that this week is an "education-free week", meaning that there will not be a lecture or a tutorial this week. Take a well-deserved brake and we continue the course in week 6!}

\section*{Week 6: Recommender systems}

\subsection*{Monday, 9--5: Lecture}
\textsc{\ding{52} Read in advance: \cite{Moller2018}.}\\
\textsc{\ding{52} Read in advance: \cite{Loecherbach2020}.}\\

In today's lecture, we will discuss different types of recommender systems. In order to understand how recommender systems work, an understanding of the concepts as discussed in previous weeks is crucial. Specifically, in order to build a recommender system, one needs to understand how to preprocess data, transfrom textual data to data matrices, and calculate similarity between texts. 

\subsection*{Tuesday, 10--5: Lab session}
During this week's lab session, we will practice with designing a recommender system yourself. You will experiment with different settings, and inspect the effects thereof on the recommendation content. 

\section*{Week 7: Text classification, part 1}
\subsection*{Monday, 16--5: Lecture}
\textsc{\ding{52} Watch: \url{https://www.youtube.com/watch?v=81vTqTz2pbM}.}\\
\textsc{\ding{52} Read in advance \cite{van_zoonen_social_2016}.}\\

In this lecture, we will take a look at text classification. We briefly review strictly rule-based methods and then we will explosure supervised machine learning.


\subsection*{Tuesday, 17--5: Lab session}
This week, you will present your group assignment. 

\subsection*{Deadline group project: Friday 20--5}
The deadline for handing in the group assignment is end of this week, \textbf{Friday 20--5 at 17:00}.

\section*{Week 8: Text classification, part 2}

\subsection*{Monday, 23--5: Lecture}
\textsc{\ding{52} Read in advance \cite{jordan_mitchell}.} \\
\textsc{\ding{52} Read in advance \cite{meppelink_reliable_2021}.}\\

In this lecture, we will take a deeper dive into supervised machine learning. We will discuss some commonly used machine learning models as well as how to validate classifiers.

\subsection*{Tuesday, 24--5: Lecture}
At the end of this last tutorial meeting, you will recieve the take-home exam. You will have time to read it in class, and ask questions if needed. \\

\subsection*{Deadline take home exam: Sunday 29--7}
Deadline of the take-home exam is \textbf{Friday 27--5 at 17:00}.







