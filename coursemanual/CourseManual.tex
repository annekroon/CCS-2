% !TEX encoding = UTF-8 Unicode

%\documentclass[a4paper,10pt]{report}
\documentclass[a4paper,10pt,twocolumn]{report}
\usepackage[natbibapa,nosectionbib,tocbib,numberedbib]{apacite}
\AtBeginDocument{\renewcommand{\bibname}{literature}}
\usepackage{color,soul}

\usepackage[utf8x]{inputenc}
\usepackage{graphicx}
\usepackage{enumerate}
\usepackage{url}

\usepackage[colorinlistoftodos]{todonotes}

\usepackage{pifont}

\usepackage{lmodern}
\usepackage{listings}
\lstset{
	basicstyle=\scriptsize\ttfamily,
	columns=flexible,
	breaklines=true,
	numbers=left,
	%stepsize=1,
	numberstyle=\tiny,
	backgroundcolor=\color[rgb]{0.85,0.90,1}
}


\let\oldquote\quote
\let\endoldquote\endquote
\renewenvironment{quote}{\footnotesize\oldquote}{\endoldquote}



\title{Computational Communication Science 2\\ Course Manual}
\author{dr. Anne Kroon (a.c.kroon@uva.nl)\\dr. Marthe Möller (a.m.moller@uva.nl) \\~\\College of Communication\\University of Amsterdam}
\date{Spring Semester 2022 (block 2)}


\begin{document}
	\maketitle
	
	\tableofcontents

	
	\chapter{About this course}
	
	This course manual contains general information, guidelines, rules and schedules for the course Computational Communication Science 2 (6 ECTs), part of the Communication in the Digital Society Minor offered by the College of Communication at the University of Amsterdam. Please make sure you read it carefully, as it  contains information regarding assignments, deadlines and grading.
	
	\section{Course description}
	
	A sales website that recommends new products to you personally, a company that uses a chatbot to answer your questions, or an algorithm that automatically identifies and warns you about fake news content: In our digital society, we use computational methods to communicate with each other every day. In this course, we will zoom in on the computational methods that lay behind these new ways of communication. We will explore the basic principles of their design, acquire an understanding of their implications, and learn how these methods can be used for science. We will work with some of these methods ourselves in the weekly tutorials to get hands-on experience with these techniques and experience their advantages and limitations first hand. During weekly lectures, we will critically discuss the role that these methods play in our daily lives and what responsibility we have when working with them. At the end of the course, you will have a basic understanding of the methods that underlie different ways of communication in the digital society, you can formulate an informed opinion about the implications of these new techniques, and you will have some first-hand experience in working with them.
	
	In this 7-week course, each week consists of one lecture that zooms in on a specific computational method and the possible applications of this method and of one tutorial in which we work with this method. Through this mixture of introductions to computational methods in the lectures and a hands-on approach during the tutorials, you will acquire knowledge on computational communication science that continues on the knowledge that you gained in CCS-2. In total, there are 28 contact hours in this course (7x 2-hour lecture and 7x 2-hour tutorial). 
	
	\section{Goals}
	Upon completion of this course, students should:
	\begin{enumerate}[a]
		\item Have a general understanding of state-of-the-art computational techniques useful to study communication phenomena in the digital society.
		\item Have a basic understanding of how to apply rule-based, unsupervised and supervised techniques to answer research questions in the field of communication science.
		\item Be able to identify key benefits and drawbacks associated with different rule-based and machine learning techniques.
		\item Have basic knowledge of what communication scientific questions can be answered using computational methods.
		\item Be able to apply a subset of these techniques independently in order to answer some basic research questions in the field of communication in the digital society.
		\item Have experience with independently solving problems in Python scripts by gathering information from online platforms.
		\item Be able to clearly communicate through written texts what steps were taken in a research project using computational methods.
	\end{enumerate}

	\section{Study materials}
	During this course, Python is the programming language that we will be working with. Hence, students should bring a laptop to each class, with a working Python environment installed.  
	
	In addition, a list of assigned readings is made available on the course Canvas page. All readings are available for download online using the UvA Digital Library or Google Scholar. If a reading is not available online, the material will be made available on the course Canvas page.
	
	
	\chapter{Rules, assignments, and grading}
	Assessment for this course is based on a mixture of individual and group assignments. 
	
	\section{Overview of assessments}
	The overall course grade is based on the following assignments:
	\begin{itemize}
		\item Regular multiple-choice questions (\(20\%\))
		\item Group assignment: Group assignment (\(20\%\))
		\item Group assignment: Group assignment: presentation (\(10\%\))
		\item Take-home exam (individual) (\(50\%\))
	\end{itemize}

	
	\section{Regular multiple-choice questions ($20\%$)}
At the start of the tutorial meeting meetings, students will be asked to answer a set of questions about that week's literature and/or the preceeding lecture. 
	
	\section{Group assignment: report ($20\%$)}
In groups of 3-4 students, you will work on a group assignment. See the description here: \url{https://github.com/annekroon/CCS-2/blob/main/week03/group-assignment.md}
	
	\section{Group assignment: presentation ($10\%$)}
In week 7, you will present your group assignment in class. 
	
	\section{Take-home exam ($50\%$)}
	During the course's last lab session, students will recieve a take-home exam. This exam will test students understanding of the concepts and literature as discussed. In addition, students are presented with a computational coding challenge on one of the techniques discussed in the course. 

	\section{Grading}
	\textcolor{red}{Students have to get a pass (5.5 or higher) for each of the following: the average mc-questions grade, the average of the group report and presentation, and for the take-home exam. If the average grade of the mc-questions or the average grade of the group assignments is lower than 5.5, a resit assignment is given to the student(s) for which they have one week to complete it. If the grade of the take-home exam is lower than 5.5, an improved version can be handed in within one week after the grade is communicated to the student. If the improved version still is graded lower than 5.5, the course cannot be completed. Improved versions of the take-home exam cannot be graded higher than 6.0.} 
	
	\section{Deadlines and submitting assignments}
	Please send all assignments and papers as a PDF file to ensure that it can be read and is displayed the same way on any device. Hardcopies are not required. Multiple files should be compressed and handed in as one .zip file or .tar.gz file. Anything exceeding a reasonable file size (approximately 5 MB) has to be send via \url{https://filesender.surf.nl/} instead of direct email. \\
	
	\textcolor{red}{Assignments that are not completed on time, will be not be graded and receive the grade 1. 
		\begin{itemize}
			\item For the group report and the take-home exam, this means that all required files need to be submitted before the deadline. 
			\item For the mc-questions, this means that the questions can only be answered during class. If a student cannot answer mc-questions due to missing class... --> tja, wat gebeurt er dan?
			\item The group presentation needs to held during the assigned class, meaning that slides or any other material used for the presentation needs to be present in class as well as at least one group member to give the presentation. 
		\end{itemize}
		Note that the deadline of an assignment is only met when the all files are submitted \emph{before} the deadline.}

	\section{Plagiarism \& fraud}
	Plagiarism is a serious academic violation. Cases in which students use material such as online sources or any other sources in their written work and present this material as their own original work without citation/referencing, and thus conduct plagiarism, will be reported to the Examencommissie of the Department of Communication Science without any further negotiation. If the committee comes to the conclusion that a student has indeed committed plagiarism the course cannot be completed. 

	General UvA regulations about fraud and plagiarism apply.

	\section{Presence and participation}
	\textcolor{red}{(Virtual) attendance is compulsory. Maximum two meetings can be missed \emph{if} both instructors are informed of this beforehand. Missing more than two meetings – for whatever reason – means the course cannot be completed.} 
	
	As you will have noticed in Computational Communication Science 1 and/or similar courses on coding, developing your skills to work with computational methods requires you to practice regularly and to be proactive when it comes to solving problems in your code. Hence, students of Computational Communication Science 2 are expected to practice with and revise the materials discussed in class at home. By practicing at home regularly in between classes, you will get the most out of this course and you will acquire the skills that you need to continue developing your programming knowledge after you finished this course.
	
	\section{Staying informed}
	It is your responsibility to check the means of communications used for this course (i.e., your email account, the course Canvas page, and the course Github page) on a regular basis, which in most cases means daily.
	
	

	

	
	
	\chapter{Course Schedule}
	
	This course is set-up in the following manner. In a regular week, on the Mondays, we have lectures. Here, key concepts are explained from a conceptual and theoretical perspective, and examples of code implementations is python are provided.  On the Tuesdays, we have lab sessions. During the lab sessions, we generally start with knowledge questions about that week's literature as well as the preceeding lecture. Afterwards, students will work on assignments that are provided. When students have finished the assignments, they can work on the group assignment. The tuturial lecture (Noon will walk around, and will help students with issues they encounter. If multiple students are encountering the same issues, these problems will be discussed plenarily. Active participation in the tutorial meetings is needed to ensure that students understand the materials, and can work on the (group) assignments.

\section*{Week 1: Course introduction \& Working with textual data}



\section*{Week 1: Course introduction \& Text-as-data}


\subsection*{Monday, 4--4: Lecture}
\textsc{\ding{52} Please read some stuff.}\\
\textsc{\ding{52} Make sure your computer is ready to use for the course (see "study materials").}\\
blablabla

\begin{itemize}
	\item{Make sure your computer is ready to use for the course (see "study materials" on Canvas)}
	\item{Read this manual and inspect the course Canvas page}
	\item{Read \cite{boumans_taking_2016}}
\end{itemize}


\subsection*{Tuesday, 5--4. Lab session}
\textsc{\ding{52} Make sure your computer is ready to use for the course (see "study materials").}\\
During our first tutorial lecture, we will start with blablabla

\section*{Week 2: Text as data}

\subsection*{Monday, 11--4: Lecture}
\textsc{\ding{52}Read in advance: chapter 10: Text as data \cite{van2021computational}.} \\
\textsc{\ding{52}Read in advance: \cite{Boumans2016}.} \\

During this meeting, we will continue with were we left off in the first week. Using some of the techniques of week 1, we will explore some simple, deductive approaches and reflect on their advantages and disadvantes. As an alternative, we will reflect on some of the benefits of inductive, bottom-up approaches. To this aim, we will discuss Bag-of-Words (BOW) representations of textual data, and discuss multiple ways of transforming text into matrices. 

\subsection*{Tuesdag, 12--4: Lab session}
In this week's lab session, we will practice with some simple top-down and bottom-up algorithm for text analysis. In order to prepare ourselves for more advanced bottom-up techniques in week 4, we will start practicing with transforming textuel to matrices, using  \texttt{sklearn}'s vectorizers. 

\section*{Week 3: Start of the group assignment}

\subsection*{Monday, 18--4: No lecture--UvA teaching free day}

\subsection*{Tuesday 19--4: Lab session}
During this tutorial meeting, we will form small groups of 3-4 students and start working on the group assignment. Specifically, we will make a start with exploring the dataset as provided, inspect relevant variables, and start working on cleaning the dataset and division of tasks. 

\section*{Week 4: Bottom up approaches to text analysis}

\subsection*{Monday, 25--4: Lecture}
\textsc{\ding{52}Read in advance chapter 11: Automatic analysis of text \cite{van2021computational}.} \\
\textsc{\ding{52}Read in advance: \cite{Brinberg2021}.} \\

In this meeting, we will discuss some prominent techniques to analyses text using bottom up techniques. Specifically, we will discuss latent dirichlet allocation, a populair approach to detect topics in textual data. In addition, we will focus on different ways to measure similarity between texts, using cosine and soft cosine similarity. 

\subsection*{Tuesdag, 26--4: Lab session}
During this tutorial meeting, we will practice with the concepts and code as introduced in Monday's lecture. We will try to measure the similarity between sets of documents using different approaches. In addition, we will practice with a simple LDA model. \\
In case you understand all concepts, and finished the in-class assignments, you can continue working on the group assignment. 

\section*{Week 5: Taking a break}

\subsection*{Monday, 2--5: No Lecture -- UvA teaching free week}

\subsection*{Tuesday, 3--5: No Lab session -- UvA teaching free week}

\textsc{\ding{52} Take a break}\\

\emph{Note that this week is an "education-free week", meaning that there will not be a lecture or a tutorial this week. Take a well-deserved brake and we continue the course in week 6!}

\section*{Week 6: Recommender systems}

\subsection*{Monday, 9--5: Lecture}
\textsc{\ding{52} Read in advance: \cite{Moller2018}.}\\
\textsc{\ding{52} Read in advance: \cite{Loecherbach2020}.}\\

In today's lecture, we will discuss different types of recommender systems. In order to understand how recommender systems work, an understanding of the concepts as discussed in previous weeks is crucial. Specifically, in order to build a recommender system, one needs to understand how to preprocess data, transfrom textual data to data matrices, and calculate similarity between texts. 

\subsection*{Tuesday, 10--5: Lab session}
During this week's lab session, we will practice with designing a recommender system yourself. You will experiment with different settings, and inspect the effects thereof on the recommendation content. 

\section*{Week 7: Text classification, part 1}
\subsection*{Monday, 16--5: Lecture}
\textsc{\ding{52} Please read some stuff.}\\
\textsc{\ding{52} Please read some stuff.}\\

\begin{itemize}
	\item Watch: \url{https://www.youtube.com/watch?v=81vTqTz2pbM}
		\item Read \cite{van_zoonen_social_2016}
\end{itemize}


\subsection*{Tuesday, 17--5: Lab session}
This week, you will present your group assignment. 

\subsection*{Deadline group project: Friday 20--5}
\textcolor{red}{The deadline for handing in the group assignment is end of this week, \textbf{Friday 20--5 at 17:00}}.


\section*{Week 8: Text classification, part 2}
\begin{itemize}
	\item{Read \cite{jordan_machine_2015}}
	\item{Read \cite{meppelink_reliable_2021}}
\end{itemize}


\subsection*{Monday, 23--5: Lecture}
\textsc{\ding{52} Read \cite{van_zoonen_social_2016}}
\subsection*{Tuesday, 24--5: Lecture}
At the end of this last tutorial meeting, you will recieve the take-home exam. You will have time to read it in class, and ask questions if needed. \\

\subsection*{Deadline take home exam: Sunday 29--7}
\textcolor{red}{Deadline of the take-home exam is \textbf{Sunday 29--5 at 17:00}}.








	
	\bibliographystyle{apacite}
	\bibliography{../literature.bib}
	
	
	
\end{document}